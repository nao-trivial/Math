\documentclass[a4paper,11pt]{article}
\usepackage[utf8]{inputenc}
\usepackage{amsmath,amsfonts,amssymb,amsthm}
\usepackage{hyperref}
\usepackage{geometry}
\geometry{margin=1in}

\title{Iteração Linear - Resumo e Exercícios}
\author{}
\date{}

\begin{document}

\maketitle

\section*{Introdução}
A \textbf{iteração linear} é uma técnica usada para encontrar aproximações para raízes de funções ou soluções de sistemas lineares. Consiste em construir uma sequência de aproximações usando uma fórmula iterativa da forma:

\[
x_{k+1} = g(x_k),
\]

onde $g(x)$ é uma função definida no intervalo de interesse e $x_k$ é a aproximação na $k$-ésima iteração.

\section*{Convergência da Iteração}
A iteração $x_{k+1} = g(x_k)$ converge para uma solução $x^*$ se:
\begin{enumerate}
    \item $x^*$ é ponto fixo de $g(x)$, ou seja, $g(x^*) = x^*$;
    \item A função $g(x)$ é contínua e diferenciável no intervalo considerado;
    \item $|g'(x)| < 1$ no intervalo de interesse.
\end{enumerate}

\textbf{Critério de Parada:} A iteração é interrompida quando $|x_{k+1} - x_k| < \epsilon$, onde $\epsilon$ é uma tolerância predefinida.

\section*{Exemplo: Método do Ponto Fixo}
Dado um problema de encontrar a raiz da equação $f(x) = 0$, podemos reescrever no $f(x) = 0$ na forma $x = g(x)$. Suponha $f(x) = x^2 - 3$. Uma escolha para $g(x)$ pode ser:

\[
g(x) = \sqrt{3}.
\]

A iteração será então:
\[
x_{k+1} = \sqrt{3}.
\]

\textbf{Nota:} Escolher uma $g(x)$ apropriada é essencial para garantir a convergência.

\section*{Exercícios}

\subsection*{Exercício 1}
Considere a equação $f(x) = x^3 - x - 1$.
\begin{enumerate}
    \item Reescreva a equação na forma $x = g(x)$ de três maneiras diferentes.
    \item Determine se as funções $g(x)$ escolhidas são adequadas para a iteração linear no intervalo $[1, 2]$.
    \item Execute a iteração linear usando uma das formas $g(x)$ com $x_0 = 1.5$ e tolência $\epsilon = 10^{-4}$.
\end{enumerate}

\subsection*{Exercício 2}
Dado o sistema linear:
\[
\begin{aligned}
3x + y &= 9, \\
x - 4y &= -6,
\end{aligned}
\]
resolva-o usando a iteração de Jacobi. Escolha $x_0 = 0$ e $y_0 = 0$ como aproximação inicial. Realize 5 iterações.

\subsection*{Exercício 3}
Para cada $g(x)$ abaixo, determine se a iteração $x_{k+1} = g(x_k)$ converge em $[0, 1]$:
\begin{enumerate}
    \item $g(x) = \frac{x^2 + 1}{2}$;
    \item $g(x) = 1 - e^{-x}$;
    \item $g(x) = \sin(x) + \frac{x}{2}$.
\end{enumerate}

\section*{Referências}
\begin{itemize}
    \item Burden, R. L., \& Faires, J. D. (2010). \textit{Numerical Analysis}. Cengage Learning.
    \item Conte, S. D., \& de Boor, C. (1980). \textit{Elementary Numerical Analysis: An Algorithmic Approach}. McGraw-Hill.
\end{itemize}

\end{document}
